\documentclass{article}
\usepackage{csquotes}

\usepackage[british]{babel}
\usepackage[backend=biber,style=apa]{biblatex}
\DeclareLanguageMapping{english}{british-apa}
\addbibresource{\jobname.bib}

\title{On the Origin of the Eukaryotes \\
           \Large\textsc{biosci \oldstylenums{210} essay}}
\author{Arman Bilge}
\date{16th September 2014}

\frenchspacing
\begin{document}

    \maketitle

    \paragraph*{Introduction}
        At the heart of systematics is the fundamental classification of an
            organism as either \emph{prokaryotic} or \emph{eukaryotic}.
        Specifically, these terms refer to the characteristics of the cell (or
            cells) that compose the organism.
        Eukaryotic cells are considered to be substantially more complex than
            their prokaryotic counterparts: some of their key features include
            a separated nucleus with carefully-organised chromasomes,
            a cytoskeleton enabling motility and holding several organelles in
            place, and a mitochondria for highly efficient transformation of
            energy, none of which have prokaryotic
            analogues~\parencite{Duv07,Kat12}.
        Naturally, one of the biggest efforts in biology is appreciating the
            evolutionary processes behind the development of the eukaryotes,
            referred to as \emph{eukaryogenesis}.
        Constructing a sound theory for eukaryogenesis is particularly
            challenging because so many different elements must be
            considered; namely, the characteristics of the last eukaryotic
            common ancestor (LECA), the biochemical processes that enabled the
            emergence of these features, and the evolutionary advantages that
            they provided enabling them to persist.
        This task is further hindered not only by the lack of a reliable fossil
            record~\parencite{Sog91}, but also the loss of molecular signal
            (i.e., DNA evidence) due to multiple substitutions at the same
            sites after some billion years of evolution and may even result in
            artefacts in phylogenetic reconstruction \parencite{Gri+10,Wil+13}.
        Note also that hypotheses regarding eukaryogenesis must also consider
            the geology and atmospheric chemistry of the earth (particularly,
            the presence of oxygen) due to its influence in the evolution of
            organelles.

    \paragraph*{Evidences regarding eukaryogenesis}
        Perhaps the most important evidence regarding the origin of the
            eukaryotes is the chimeric nature of the eukaryotic genome, meaning
            that the genome is a mosaic of genes from different sources,
            primarily Euryarchaeota (an archaebacteria) or Alphaproteobacteria
            (an eubacteria) \parencite{Kat12}.
        Furthermore, there is a clear pattern to this mosaic: informational
            genes tend to be derived from archaebacterial sources while
            operational genes are derived from eubacteria \parencite{Duv07}.
        There is also strong evidence, both morphological and molecular, that
            the mitochondria is derived from an Alphaproteobacteria;
            specifically, one closely related to \emph{Rickettsia}
            \parencite{Duv07,Kat12}.
        However, the mitochondrial genome is substantially reduced compared to
            that of its original form.

    \paragraph*{Theories of eukaryogenesis}
        Hypotheses regarding the origin of the eukaryotes may generally be
            placed in a few categories that attempt to explain how the
            eukaryotic genome acquired its chimeric nature.
        The first considers the phylogenetic relationship of the eukaryotes to
            the eubacteria and the archaebacteria.
        The three domain model places the eukaryotes as sister to the
            archaebacteria, whereas the two domain, or eocyte, model places the
            eukaryotes within the archaebacteria \parencite{Wil+13}.
        The endosymbiotic theory of eukaryotes predicts that the mitochondria
        originated as a symbiotic bacterium that was engulfed by the ancestral,
        proto-eukaryote.
        Alternative theories, such as the ring of life hypothesis, predict a
            fusion event between an archaebacteria and an eubacterium
            \parencite{RL04}.

    \paragraph*{Evidence for a complex proto-eukaryote}

    The accumulation of recent evidence points to a complex proto-eukaryote
        descended directly from the archaebacteria that engulfed an
        endosymbiotic Alphaproteobacteria as one of the last steps before
        becoming LECA.
    In contrast with older studies \parencite{Hed+01,Gri+10}, more recent work yields
        substantially greater support to the two domain, eocyte theory
        \parencite{KWG11,Thi+12,Wil+13}
    Evidence for the complexity of the proto-eukaryote includes the estimation
        of not only a large number of genes \parencite{Koo10a} but also the
        substantial ocurrence of introns \parencite{CRK11} in its genome.
    These genes indicate an organism capable of both capable of flagellar and
        amoeboid movement and a flexible energy conversion system
        \parencite{Koo10a}.
    Because most of the eukaryotic organelles could have developed anaerobically
        \parencite{Duv07}, the protoeukaryote was probably easily capapble of
        engulfing objects from its surrounding.
    This is viable from an evolutionary standpoint because the ability to
        engulf objects, such as food, makes the organism independent from
        residing on an edible medium \parencite{Duv07}.
    It then follows that the mitochondria was engulfed endosymbiotically 1.8
        billion years ago, creating LECA and enabling it to perform aerobic
        respiration.
    It is also highly likely that LECA may have taken up genetic material from
        the organisms that it preyed on, further accounting for the chimeric
        nature of its genome \parencite{Kat12}.
    The discovery of a virus-like gene transfer agent in Alphaproteobacteria
        also accounts well for the hypothetical transfer of genes from the
        ancestral mitochondria into the eukaryotic genome \parencite{RA11}.

    As \textcite{Gri+10} indicate, new computational methods that can handle
        complex genome evolution (i.e., lateral gene transfer), such as the
        recent method developed by \textcite{Sjo+14} will likely be important
        to resolving the phylogenetic history of the eukaryotes, particularly
        given the tendency for different methods to resolve differing histories
        for even the same data \parencite{Wil+13}.

    \printbibliography

\end{document}
