\documentclass{article}
\usepackage{amsmath}
\usepackage{amssymb}
\usepackage{booktabs}
\usepackage{csquotes}

\usepackage[backend=biber,style=alphabetic]{biblatex}
\addbibresource{h-w.bib}

\usepackage{acronym}
\acrodef{BF}{Bayes factor}
\acrodef{HWE}{Hardy--Weinberg equilibrium}

\usepackage{caption}
\captionsetup{font=small,labelfont=bf,labelsep=period}

\title{Mendelian Inheritance at the Population Level and the Hardy--Weinberg
           Model \\
       \Large\textsc{biosci \oldstylenums{210} lab report}}
\author{Arman Bilge}
\date{August 21, 2014}

\frenchspacing
\begin{document}

    \maketitle

    \section*{Introduction}

        The characterisation of Mendelian inheritance was an important step
            forward in evolutionary theory because it provided the
            foundations of a physical mechanism for creating variation between
            individuals (that evolutionary processes could then act upon).
        Gregor Mendel, whose work on garden peas initiated understanding of
            Mendelian inheritance, proposed three laws describing how traits
            are passed from parents to offspring; namely, the laws of
            segregation, independent assortment, and dominance.
        However, Mendel's laws apply only at the individual level, but
            evolutionary processes act on populations as a whole, indicating a
            need for a cohesive theory of inheritance at the level of a
            population.

        The Hardy--Weinberg model applies the concepts of Mendelian
            inheritance, specifically the law of segregation, to predict the
            frequencies of particular genotypes within an idealised population.
        Specifically, the model states that when the frequencies of some
            alleles $A$ and~$a$ in the population are $p$ and~$q$,
            respectively, then the frequencies of the genotypes $AA$, $Aa$,
            and~$aa$ are $p^2$, $2pq$, and~$q^2$, respectively.
        In its simplest form, the Hardy--Weinberg model is applicable only to
            diallelic genetic loci in populations of sexual, diploid organisms.
        Furthermore, it assumes that there is no migration, mutation, or
            selection and that the population is infinitely large with
            non-overlapping generations.
        However, these \enquote{limitations} of the model in fact make it a
            powerful tool for statistically testing hypotheses regarding a
            locus in a population.
        Rejecting a null hypothesis that a gene is in \ac{HWE} suggests that
            at least one of the assumptions is invalid and prompts further
            scientific inquiry.

        To learn about Mendelian inheritance at the population level, we
            exercised a simple simulation with approximately 60 individuals,
            each with two loci, blue and green, (plus a third locus for sex
            determination) over 7 generations.
        In the initial generation, each locus had only two possible alleles.
        Random mating was then carried out for three generations.
        In the fourth generation, a mutant allele was introduced at both loci.
        While the mutant allele $G1$ at the green locus rendered no advantage,
            the mutant allele $B1$ at the blue locus gave its carrier a greater
            chance of survival.
        Normal alleles within the population were replaced at random by mutant
            alleles, 5 each for the blue and green loci, followed by another
            three generations of random mating.
        I predicted that the $B1$ mutant allele would propagate given the
            selective pressure on it, whereas the $G1$ mutant would remain at a
            relatively constant frequency, subject only to genetic drift.

    \section*{Data and Analysis}

    Data were collected of the frequency of each genotype at the blue and green
        loci when there were only two possible alleles for each; after
        introducing the mutant alleles, only per-allele frequencies were
        noted due to the increased number of possible
        genotypes~(table~\ref{tab:data}).

    \begin{table}
        \centering
        \caption{Expected and observed allele and genotype counts at blue and
                     green loci.
                 Expected values were computed under the non-exact
                     Hardy--Weinberg model and rounded to the nearest tenth.
                 The exact Hardy--Weinberg model was considered the null model
                     when calculating Bayes factors.}
        \label{tab:data}
        \begin{tabular}{l r r r r r r r r}
            \toprule
            & \multicolumn{4}{c}{\textbf{Blue}}
                & \multicolumn{4}{c}{\textbf{Green}} \\
            \emph{Generation} & 0 & 3 & 4 & 7 & 0 & 3 & 4 & 7 \\
            \midrule
            Light allele & 90 & 88 & 87 & 82 & 75 & 75 & 73 & 83 \\
            Dark allele & 34 & 34 & 36 & 32 & 47 & 47 & 46 & 36 \\
            Mutant allele & 0 & 0 & 5 & 6 & 0 & 0 & 5 & 4 \\
            \midrule
            \textsc{observed} \\
            Homozygous light & 33 & 32 & & & 35 & 23 \\
            Heterozygous & 24 & 24 & & & 5 & 29 \\
            Homozygous dark & 5 & 5 & & & 21 & 9 \\
            \midrule
            \textsc{expected} \\
            Homozygous light & 32.7 & 31.7 & & & 23.1 & 23.1 \\
            Heterozygous & 24.7 & 24.5 & & & 28.9 & 28.9 \\
            Homozygous dark & 4.7 & 4.7 & & & 9.1 & 9.1 \\
            \midrule
            Bayes factor & 0.261 & 0.261 & & & $3.78 \times 10^9$ & 0.295 \\
            \bottomrule
        \end{tabular}
    \end{table}

    In light of criticisms of $\chi^2$ tests for \ac{HWE}~\cite{WCA05},
        I employed an exact Bayesian statistical approach~\cite{CMV11} in its
        place (where \enquote{exact} indicates a relaxation of the infinite
        population assumption).
    Bayesian hypothesis testing considers the ratio of the probability of the
        data given the alternative model~$M_1$ to that given the null
        model~$M_0$ (in this case, \ac{HWE})~\cite{Wak10}.
    This value is termed the \ac{BF} and formally defined as
    \begin{equation}
        \text{BF} = \frac{\mathbb{P}\left(D\mid M_1\right)}
                         {\mathbb{P}\left(D\mid M_0\right)}.
    \end{equation}
    Note that a \ac{BF} substantially larger than 1 provides strong evidence
        to reject the null model in favour of the alternative.
    I computed \ac{BF}s using the \texttt{R}~\cite{RCT14} package
        \texttt{HWEintrinsic}~\cite{Ven12}, which considers the full (or
        saturated) model as the alternative~\cite{CMV11}
        (table~\ref{tab:data}).

    For generation~0, the failure to reject the null model at the blue locus
        suggests that it was in \ac{HWE}. However, the null model was
        decisively rejected at the green locus, indicative that it was not in
        \ac{HWE}.
    Three generations later, the statistical tests failed to reject the null
        model at both the blue and green loci, suggesting that both were in
        \ac{HWE}.
    At both loci, the allele frequencies remained constant as did the genotype
        frequencies at the blue locus (although it appears that a homozygous
        light individual went missing).
    However, at the green locus, the genotype frequencies changed dramatically,
        correlating with the statistical evidence that between the generations
        the population went from non-\ac{HWE} to \ac{HWE}.
    In general, the deviation of the observed genotype frequencies with regard
        to the frequencies as expected under \ac{HWE} was directly correlated
        with the statistical support for the model.
    When the population was in \ac{HWE} at a particular locus, the observed and
        expected frequencies differed by no more than~$\pm1$ individual.

    However, after adding the mutant alleles to the gene pool, there was less
        overall stability in the frequencies of genotypes.
    Although modern implementations of tests for \ac{HWE} can consider more
        than two alleles per loci~\cite{Wak10}, unfortunately we did not
        collect the data necessary to enable these calculations.
    Therefore, it is unclear whether or not they were in \ac{HWE} at any point.

    \section*{Discussion}

        Overall, this simulation activity showed strong evidence validating the
            Hardy--Weinberg model in a (finite), idealistic population.
        As predicted by the theory, under random mating, allele frequencies
            remained constant and genotype frequencies converged to \ac{HWE}.
        Interestingly, even a locus that began not in \ac{HWE} was in \ac{HWE}
            after three generations, providing additional support for the
            previous statement.
        We also observed how this equilibrium may be disrupted, by introducing
            elements to the simulation that violate the assumptions of
            \ac{HWE}.
        The effects of genetic drift were more readily apparent with three
            possible alleles per locus, most likely due to the greater number
            of possible genotypes while the population itself did not grow in
            size.
        This was best illustrated by the green locus, where the mutant allele
            was selectively neutral, yet the frequencies of all three possible
            alleles changed over three generations.

        We may question the plausibility of the initial state of the
            population, generation~0, for which the blue locus was in
            \ac{HWE} but the green locus was not.
        In an ideal population, obviously this would be possible; however,
            given Mendel's law of independent assortment, it is viable for one
            locus to be out of \ac{HWE} irregardless of other the state of the
            other loci.
        Ultimately, for the purposes of our simulation, the effect was likely
            negligible, because the green locus eventually ended up in \ac{HWE}
            and the blue locus remained in \ac{HWE}.

        In the dataset we observed little, if any, evidence of the effects of
            selection on alleles, contradictory to my hypothesis.
        I am unsurprised that changes in the frequency of the $B1$ mutant
            allele after three generations were minimal despite the apparent
            selective pressure on it.
        It is important to note that individuals carrying the $B1$ allele had
            an advantage only over their siblings; that is, under the rules of
            our simulation, two parents must always replace themselves with two
            children, irregardless of whether either carried the advantageous
            $B1$ allele.
        Therefore, an individual carrying $B1$ had no greater reproductive
            success than a non-sibling that does not carry $B1$.
        I expect that in a simulation where an individual carrying an allele
            with an advantageous mutation has a clear advantage over all
            individuals then the effect of selection on alleles would be
            demonstrated more strongly.

    \printbibliography

\end{document}
