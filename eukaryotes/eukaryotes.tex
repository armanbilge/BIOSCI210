\documentclass{article}
\usepackage{csquotes}
\usepackage{paralist}

\usepackage[british]{babel}
\usepackage[backend=biber,style=apa]{biblatex}
\DeclareLanguageMapping{english}{british-apa}
\addbibresource{\jobname.bib}

\title{On the Origin of the Eukaryotes \\
           \Large\textsc{biosci \oldstylenums{210} essay}}
\author{Arman Bilge}
\date{16th September 2014}

\frenchspacing
\begin{document}

    \maketitle

    At the heart of systematics is the most fundamental classification of an
        organism as either \emph{prokaryotic} or \emph{eukaryotic}.
    Specifically, these terms refer to the characteristics of the cell (or
        cells) that compose the organism.
    Eukaryotic cells are considered more complex than prokaryotic cells and
        contain organelles, such as a nucleus and mitochondria, as well as a
        sophisticated internal structure, the cytoskeleton
        \parencite{Koo10b,Kat12}.
    However, our interests as biologists extend beyond simply the differences
        between these cell types to appreciating the evolutionary processes
        that led to the development of the eukaryotic cell.
    This enigma may be broken into three closely-intertwined problems, which
        consider\begin{inparaenum}[\itshape a\upshape)]
        \item the features of the last eukaryotic common ancestor (LECA);
        \item the biochemical processes that enabled the emergence of these
        features; and
        \item the evolutionary advantages of these features provided that enabled them
        to persist.
    \end{inparaenum}
    Unfortunately, there exist a plethora of problems that make this such a
    challenging task.
    Any coherent theory for eukaryogenesis must consider not only the previous concepts, but also the temporal scale on which
    characteristics evolved and (implicitly) the order which they appeared.
    Furthermore, geochemical evidence also becomes important when considering
    the evolution of biochemical pathways (e.g., those involving oxygen).
    This process is hindered by a lack of reliable evidence in the fossil
    record \parencite{Sog91} and varying rates of evolution across prokaryotes
    and eukaryotes \parencite{Hed+01}.

    Despite these difficulties, there are several competing models for
    various aspects of eukaryogenesis. One set of hypotheses regards the
    evolutionary relationship between the bacteria, archaea, and eukaryotes.

    \printbibliography

\end{document}
